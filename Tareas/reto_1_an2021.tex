%--------------------------------------------------------------------
% Formato para talleres y quices
% Eddy Herrera Daza M.Sc.
% herrera.eddy@gmail.com
%eherrera@javeriana.edu.co
%--------------------------------------------------------------------
%--------------------------------------------------------------------
\documentclass[12pt,letterpaper]{exam}
\usepackage[left=2cm,top=2cm,right=2cm,bottom=4cm]{geometry}
\usepackage{hyperref}
\usepackage[utf8]{inputenc}
\usepackage[utf8]{inputenc} %tildes por teclado
\usepackage[spanish,activeacute]{babel} % Escribir en espanol
\decimalpoint
\usepackage{enumerate}
\usepackage{eurosym}
\usepackage{latexsym,amsmath,amsthm,amssymb,amsfonts,bbm, dsfont}
\usepackage[mathscr]{euscript}
\usepackage{ae,aecompl}
\usepackage{graphicx}
\usepackage{fancybox}
\DeclareGraphicsExtensions{.pdf,.png,.jpg} %solo para PDFLaTeX
\usepackage{float} %para tener manejo de ubicacion de tablas y graficas
\usepackage{subfigure} %varias figura en una
% Paquete para generar varias columnas y filas
\usepackage{multicol}
\usepackage{multirow}
%----modificar las caption-----
\usepackage[font=small,labelfont=small,labelfont=bf,textfont=it]{caption}
%--------------------------------------------------------------------
\newcommand{\base}[1]{\underline{\hspace{#1}}}
%--------------------------------------------------------------------
\newcommand{\uni}{Análisis Numérico}
\newcommand{\fac}{Primer Reto }
\newcommand{\dep}{Eddy Herrera Daza }
%\newcommand{\mat}{Estadística II} %Materia
%\newcommand{\tema}{Pruebas de Hipotesis} %Tipo y Número de Quiz
%\newcommand{\autor}{Eddy Herrera Daza}
%%\newcommand{\fecha}{Mayo 2017}
%\newcommand{\espacio}[1]{\vspace{#1}}
%---------------------------------------------------------------------
\boxedpoints
\pointname{}
\bonuspointname{(bono)}
\extrawidth{0.8in}
%\extrafootheight{1.25ino
\extraheadheight{-0.1in}
\pagestyle{headandfoot}
\footrule
\headrule
\firstpageheader{}{}{}
\firstpagefooter{}{\thepage $\,$ de \numpages}{}
\runningfooter{\uni}{\thepage $\,$ de \numpages}{}

\setlength{\columnsep}{1.5cm}
\setlength{\columnseprule}{0.5pt}   % default=0pt (no line)

%---------------------------------------------------------------------------
\begin{document}
	
\begin{tabular}{lr}
    \multirow{2}{*} {\includegraphics[height=1.5cm]{logo}}
    	& \hspace{0.5 cm} {\textbf{\uni}} \\	
    	& {\textbf{\fac}} \\
    	& {\textbf{\dep}} 
%    	& {\textbf{\mat: \tema}} 
\end{tabular}\\
\base{19.5cm}
%\textbf{Pruebas de Hipótesis } \\

%cuerpo del documento


%\begin{center}
%\textbf{Recomendaciones}
%\end{center}
%\begin{enumerate}[$\bullet$]
%\item No se resuelven preguntas del contenido a evaluar
%\item No se permite el uso del celular. No se permite el uso de formulas, ejemplos, tablas material no autorizado
%\item El uso del equipo de computo es personal e intransferible.
%\item \underline{Tiene una duración máxima de $90$
% \textit{minutos}}
%\end{enumerate}
%\base{19.5cm}\\ \\

%\begin{center}
%\shadowbox{Estadistica}
%\end{center}



\section{Algoritmo Brent }
Este algoritmo de Brent, utiliza en cada punto lo más conveniente de las estrategias del de la bisección y del de la secante (o Muller). Este método suele converger muy rápidamente a cero; para las funciones difíciles ocasionales que se encuentran en la práctica.\\
\textbf{Problema}:Aplicar el algoritmo de Brent para encontrar las raíces del polinomio, con un error menor de $2^{-50}$:
\begin{equation}
	f(x)=x^{3}-2x^{2}+4x/3-8/27
\end{equation}

\section{Intersección entre curvas}
La intersección entre dos curvas es un problema comun del cálculo y que enfrenta el desafio de solucionar un sistema de ecuaciones no lineales. No obstante, la solución se puede encontrar reduciendo el problema a determinar una aproximación adecuada de los ceros de una ecuación, con un error menor de $2^{-16}$ .\\

\textbf{Problema}: Aplicar la  técnica de aproximación a la raíz, que desarrollo en el trabajo en grupo, para encontrar la intersección entre 
\begin{equation}
	x^{2}+xy=10;y+3xy^{2}=57
\end{equation} 
 

\section{Librerías en R y/o Python}

El uso adecuado de las herramientas numéricas en Pyrhon o R es importante en cualquier problema.Es por esto, que es necesario revisar su uso e impklementación para la solución de un problema.\\ \textbf{Problema}:Revisar las librerias numpy y SciPy en Python para  resolver los problemas anteriores.\\
\textbf{Problema}:En el caso de utilizar R, revisar la función base polyroot,la función uniroot basado en el algoritmo de Brent en la base de R y los paquetes pracma y rootSolve, para resolver los problemas anteriores. 


\end{document}
\vspace{0.1in}
%--------------------------------------------------------------------------------------------------------------------%

%--------------------------------------------------------------------------------------------------------------------%

%\item

%\vspace{0.1in}
%--------------------------------------------------------------------------------------------------------------------%

%--------------------------------------------------------------------------------------------------------------------%


